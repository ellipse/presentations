\documentclass{beamer}

\mode<presentation>
{
  \usetheme{Warsaw}
  \setbeamercovered{transparent}
}
\usepackage{graphicx}
\usepackage{xcolor}

\usepackage[english]{babel}

\usepackage[latin1]{inputenc}

\usepackage{times}
\usepackage[T1]{fontenc}

\title[Solar Power Satellites]{
  Progress in Microwave Engineering
}
\subtitle{
  Microwave Power Transmission in Solar Power Satellites
}
\author[xdgxygpl@gmail.com]{
  Guo~Xiangyu 02109051\and
  Bao~Feng 02109003\and
  Zhao~Liang 02109033
}
\institute[SEE at XDU]{
  School of Electronic Engineering\\
  Xidian University
}
\date[2013]{
  May 2013
}
\subject{Microwave Engineering}

\pgfdeclareimage[height=0.5cm]{university-logo}{xdlogo.pdf}
\logo{\pgfuseimage{university-logo}}


% uncover everything in a step-wise fashion
%\beamerdefaultoverlayspecification{<+->}

\begin{document}
%\includeonlyframes{bwdth}
\begin{frame}
  \titlepage
\end{frame}

\begin{frame}[label=outl]{Outline}
  \tableofcontents
  % [pausesections]
\end{frame}

\section{Introduction}

\subsection{What are SPS?}

\begin{frame}[label=what1]{SPS:Solar Power Satellites}
  \begin{definition}
    A \alert{Solar Power Satellite} is \pause
    a satellite which can be used to harvest solar energy in space
    and to transmit it to the ground.
  \end{definition}
  \pause
  \begin{block}{Main features}
    \begin{itemize}
    \item Located in a \emph{geosynchronous orbit}.
      (Height:$3.6\times 10^5$ km; Linear Velocity:$3.07$ km/s)
    \item Transmitting energy using microwave.
      (Frequency: 5.8 GHz)
    \end{itemize}
  \end{block}
\end{frame}

\begin{frame}[label=what2]{A complete SPS system}
  \begin{columns}
    \column{.5\textwidth}
    {\small A complete SPS system consists of a flight segment
      ``\emph{solar power satellite}'' and a ground segment
      ``\emph{rectenna}''.\cite{MPTtech4SPS}}
    \column{.5\textwidth}
    \begin{figure}[H]
      \centering
      \includegraphics[width=.8\textwidth]{SPS1.png}
      \caption{Configuration of the SPS system.\cite{MPTtech4SPS}}
    \end{figure}
  \end{columns}
\end{frame}


\subsection{Why SPS?}
\begin{frame}[label=why1]{Main advantages}
  {compared to traditional terrestrial method}
  \begin{columns}
    \column{.6\textwidth}
    \begin{itemize}
      
    \item<2-> Environment friendly.
    \item<3-> Unaffected by weather conditions.
    \item<4-> Much larger average solar power available per unit area.
    \item<5-> Much longer time in view of the sun.
    \end{itemize}

    \column{.4\textwidth}
    \begin{figure}[H]
     \centering
      \only<5->{\includegraphics[width=0.85\textwidth]{SPS4.eps}}
      \only<5->{\caption{Nearly 24-hour operative.}}
    \end{figure}
  \end{columns}
\end{frame}


\subsection{A Brief History of SPS}
\begin{frame}[label=his]{History}
  \begin{columns}
    \column{.5\textwidth}
    \begin{itemize}
      
      \uncover<2->{\item (a) Pioneering work by Nikola Tesla
        (early 20th century).}
      \uncover<3->{\item (b) NASA JPL demonstration (1975)}
      \uncover<4->{\item (c) Microwave transmission experiment in space
        (1983,1993)}
    \end{itemize}
    \column{.5\textwidth}
    \begin{figure}[H]
     \centering
      \includegraphics[angle=90,width=.67\textwidth]{SPS5.png}
      \caption{Historical milestones.\cite{MPTtech4SPS}}
    \end{figure}
  \end{columns}
\end{frame}


\section{Key Points to Implementation}
\begin{frame}[label=spsovv]{Overview of a SPS system}
  \begin{block}{}
    An SPS system includes a solar cell array, magnetrons, circularly
    polarized(CP) phased arrays, and a CP rectenna array.[2]
  \end{block}

  \begin{figure}[H]
    \centering
    \includegraphics[width=\textwidth]{SPS6.png}
    \caption{SPS system diagram.\cite{MPTHMSP}}
  \end{figure}
\end{frame}


\subsection{Transmitting Terminal}

\begin{frame}[label=sc]{Solar cell}
  \begin{block}{}
  Technically, this is not what we are concerned.
  But it do become the bottleneck of the SPS system.
  \end{block}
  The efficiency of solar sunlight to dc is listed below \pause
  \begin{description}
    \item[Practical tech] \hfill \\
      Around only 30\%.\pause
    \item[Adjustable Spectrum Lattice Matched] \hfill \\
      Up to 44\% (\emph{Solar Junction} in Oct, 2012).\pause
    \item[Quantum Dot Tech] \hfill \\
      Theoretical maximum value is 75\%,
      but cannot be achieved in a short term.\pause
  \end{description}
\end{frame}

\begin{frame}[label=bwdth]{For the transmitting antenna}
  \begin{block}{Requirement}
    With a height of $3.6\times 10^7$ m, we need a extremly narrow
    beamwidth to
    \pause
    \begin{itemize}
    \item reduce sidelobes and spillover losses, \pause
    \item minimize the terrestrial rectenna's size.
    \end{itemize}
  \end{block}
  \begin{figure}[H]
    \centering
    \only<3>{\includegraphics[width=0.7\textwidth]{SPS7.eps}}
  \end{figure}
\end{frame}

\begin{frame}[label=pharray]{Solution: Phased Array}
  \begin{block}{An Example}
    An array composed of $9\times 9$ subarrays, with each subarray
    contains $8\times 8$ antenna elements.
  \end{block}
  \begin{figure}[H]
    \centering
    \includegraphics[width=0.8\textwidth]{pharray.png}
    \caption{\footnotesize (a)amplitude taper;
      (b)resultant array patterns.\cite{MPTHMSP}}
  \end{figure}
\end{frame}

\subsection{Retrodirectivity}
\begin{frame}[label=retroant]{What's Retrodirective Antenna and Why?}
  \begin{definition}
    A \alert{retrodirective antenna} is an antenna that transmits
    signal back in the same direction it came from.
  \end{definition}
  \pause
  Also due to the height of the satellite, even the slightest error
  of the transmitting phased array would results in a huge deviation
  on the ground.
  \pause
  \begin{figure}[H]
    \centering
    \only<3>{\includegraphics[width=.8\textwidth]{retrodirect.png}}
    \only<3>\caption{\footnotesize A feedback loop guided by
      a pilot beam.\cite{MPTtech4SPS}}
  \end{figure}
\end{frame}

\begin{frame}[label=retroRA]{Recent Advances}
  Japan scientists has developed a system with accuracy of
  \alert{0.4 degrees RMS} in 2011.\cite{PAandRA4MPT}
  \begin{columns}
    \column{.5\textwidth}
    \begin{figure}[H]
      \centering
      {\includegraphics[width=1.1\textwidth]{RA1.png}}
      \caption{\footnotesize Software retro-directive
        system.\cite{PAandRA4MPT}}
    \end{figure}
    \column{.5\textwidth}
    \begin{figure}[H]
      \centering
      {\includegraphics[width=1.1\textwidth]{RA2.png}}
      \caption{\footnotesize Step-tracking system.\cite{PAandRA4MPT}}
    \end{figure}
  \end{columns}
\end{frame}

\subsection{Rectenna}
\begin{frame}[label=rect]{Functionality}
  \begin{definition}
    The \alert{rectenna} is used to capture and convert
    RF energy to dc power.
  \end{definition}
  \begin{figure}[H]
    \centering
    {\includegraphics[width=0.9\textwidth]{rectenna.png}}
    \caption{\small (a)Rectenna schematic; (b) Photograph of an
    LP rectenna.\cite{MPTHMSP}}
  \end{figure}
\end{frame}
\begin{frame}[label=rectRA]{Recent Advance}
  A 5.8 GHz rectenna achieving 68.5\% conversion efficiency has been
  developed at the University of Hong Kong.\cite{DESRECT}
  \begin{figure}[H]
    \centering
    {\includegraphics[width=0.65\textwidth]{uhk_rect.png}}
    \caption{\footnotesize University of Hong Kong
    designed.\cite{MPTHMSP}}
  \end{figure}
\end{frame}
\section{Roadmap}

\begin{frame}[label=roadmap]{Roadmap for commercial SPS}
  {\footnotesize Three phases: \emph{research}, \emph{development}, and
  \emph{commercial} phase}
  \begin{figure}[H]
    \centering
    \includegraphics[width=0.8\textwidth]{roadmap.png}
  \end{figure}
\end{frame}


\appendix
\begin{frame}[label=citation]
  \frametitle<presentation>{Bibliography and Further Reading}
    
  \begin{thebibliography}{10}
     
    \footnotesize
    \beamertemplatearticlebibitems
    % Followed by interesting articles. Keep the list short. 

  \bibitem{MPTtech4SPS}
    S.~Sasaki, K.~Tanaka, and K.~Maki,
    \newblock Microwave Power Transmission Technologies for Solar
    Power Satellites.
    \newblock {\em Proceedings of the IEEE}, Vol.101, No.6,
    June 2013.

  \bibitem{MPTHMSP}
    B.~Strassner, II, and K.~Chang,
    \newblock Microwave Power Transmission: Historical Milestones
    and System Components
    \newblock {\em Proceedings of the IEEE}, Vol.101, No.6,
    June 2013.

  \bibitem{PAandRA4MPT}
    Y.~Homma, T.~Sasaki, K.~Namura, F.~Sameshima, T.~Ishikawa,
    H.~Sumino, and N.~Shinohara,
    \newblock New Phased Array and Rectenna Array Systems for
    Microwave Power Transmission Research
    \newblock {\em IMWS-IWPT2011}, 12-13, May 2011.

  \bibitem{DESRECT}
    C. Chin, Q. Xue, and C. Chan,
    \newblock Design of a 5.8-GHz rectenna incorporating a new
    patch antenna
    \newblock {\em IEEE Antennas and Wireless Propag. Lett.},
    Vol. 4, pp. 175-178, 2005.
  \end{thebibliography}
\end{frame}

\end{document}


